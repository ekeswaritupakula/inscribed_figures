\documentclass[10pt]{beamer}
\usefonttheme{serif}
\usepackage{bookman}
\usepackage{hyperref}
\usepackage[T1]{fontenc}
\usepackage{graphicx}
\usecolortheme{orchid}
\beamertemplateballitem
% set colors
\usefonttheme{professionalfonts}
\usepackage{natbib}
\usepackage{hyperref}
%-----------------------------------------------------------
\setbeamerfont{title}{size=\large}
\setbeamerfont{subtitle}{size=\medium}
\setbeamerfont{author}{size=\medium}
\setbeamerfont{date}{size=\medium}
\setbeamerfont{institute}{size=\medium}
\title[]{INSCRIBED FIGURES}
\author[]{P.Sravanthi: 21B01A5486: AIDS-B\\ \vspace{4} Sk.Kareena: 21BO1A0249:
EEE\\ \vspace{4}B.kalyani: 21B01A1218: IT-A\\ \vspace{4} T.Ekeswari:
21B01A12I2: IT-C\\ \vspace{4} S.Bhanu Sri: 22B05A1216: IT-C}
\date[]
{MARCH 25, 2023}
%------------------------------------------------------------
%This block of commands puts the table of contents at the
%beginning of each section and highlights the current section:
%\AtBeginSection[]
%{
% \begin{frame}
% \frametitle{Contents}
% \tableofcontents[currentsection]
% \end{frame}
%}
\AtBeginSection[]{
\begin{frame}
\vfill
\centering
\begin{beamercolorbox}[sep=8pt,center,shadow=true,rounded=true]{title}
\usebeamerfont{title}\insertsectionhead\par%
\end{beamercolorbox}
\vfill
\end{frame}
}
%------------------------------------------------------------
\begin{document}
%The next statement creates the title page.
\frame{\titlepage}
%------------------------------------------------------------
\begin{frame}{INTRODUCTION: }
\begin{itemize}
\item Given a square of length 'n' and four circles of equal radius 'r' inscribed in it where
each circle is located in each corner and tangent to two sides of the square. we have to
calculate the area of the largest circle or square that lies completely within the large square and
intersects all the four circles in atmost four points.
\end{itemize}
\end{frame}
\begin{frame}{If the shape is circle:}
\includegraphics[width=0.3\linewidth]{circle_inscribed.png}
\begin{wrapfigure}{\hspace{10} }{\textwidth}
\caption{}
\label{fig:wrapfig}
\end{wrapfigure}
\begin{itemize}
\item \\
The distance from the corner of square to centre of circle,
OA = DE = √2 r \\ By Pythagoras theorem, \\ OE² = OF² + EF² \\
OE = √( l² + l² ) \\
OA + AB + BC + CD + DE = √2l
\\ √2r + r + x + r + √2r = √2l \\
x + 2r + 2√2r = √2l \\
x + 2r(1+ √2) = √2l \\
x = √2l - 2r(1+ √2) \\
Here x is the diameter of the required circle \\
Area of the required circle, \\
\hspace{17} = 3.14 × ( x/2 ) × ( x/2 )
\end{itemize}
\end{frame}
\begin{frame}{If the shape is square:}
\includegraphics[width=0.3\linewidth]{square_inscribed.png}
\begin{wrapfigure}{\hspace{10}}{\textwidth}
\caption{}
\label{fig:wrapfig}
\end{wrapfigure}
\begin{itemize}
\item \\
The distance from the corner of square to centre of circle, \\ x=√2 r \\ By Pythagoras theorem, \\
OE² = OF² + EF² \\
OE = √( l² + l² ) \\
OA + AB + BC + CD + DE = √2l
\\ √2r + r + x + r + √2r = √2l \\
x + 2r + 2√2r = √2l \\
x + 2r(1+ √2) = √2l \\
x = √2l - 2r(1+ √2) \\
Here x is the length of diagonal of the required square \\
Area of the required square = ( x / \sqrt{2} ) × ( x / \sqrt{2} ) \\
\end{itemize}
\end{frame}

\begin{frame}{PACKAGES:}
\begin{itemize}
\item Math Module: we have imported math module in this project. The math module is a
built in module that contains a set of math methods and constants. we have used the math
method "math.sqrt()" to return the square root of a number and the math constant "math.pi" to
return the value of pi.
\end{itemize}
\begin{itemize}
\item Sys Module: we have imported sys module and used it's variable "sys.argv" which is
a list of command line arguments. In this project, length, radius and shape are passed as
command line arguments.
\end{itemize}
\end{frame}
\begin{frame}{CHALLENGES:}
\begin{itemize}
\item If the given diameter of the circle exceeds the
given length of the large square, then all the circles get overlapped and there will be no
inscribed circle or square that intersects all the four circles in atmost four points.
\end{itemize}
\begin{itemize}
\item To overcome this, we have given the condition that the diameter of the circle should
be less than the length of the large square
\end{itemize}
\end{frame}
\begin{frame}{STATISTICS: }
\begin{itemize}
\item Number of lines of code : 24
\end{itemize}
\begin{itemize}
\item Number of functions : 3 \\ \vspace{3}
\hspace{5} areaOfCircle( ) \\ \vspace{1} \hspace{5} areaOfSquare( ) \\ \vspace{1} \hspace{5}
inscribedfigures( )
\end{itemize}
\end{frame}
\begin{frame}{PROGRAM:}
\includegraphics[width=1.0\textwidth]{code.png}
\end{frame}
\begin{frame}{OUTPUT:}
\includegraphics[width=1.0\textwidth]{output.png}
\end{frame}
\begin{frame}
\textcolor{myNewColorA}{\Huge{\centerline{Thank you!}}}
\end{frame}
\end{document}
